% LaTeX mintafájl szakdolgozat és diplomamunkáknak az
% SZTE Informatikai Tanszékcsoportja által megkövetelt
% formai követelményeinek megvalósításához
% Modosítva: 2011.04.28 Nemeth L. Zoltan
% A fájl használatához szükséges a magyar.ldf 2005/05/12 v1.5-ös vagy későbbi verziója
% ez letölthető a http://www.math.bme.hu/latex/ weblapról, a magyar nyelvű szedéshez
% Hasznos információk, linekek, LaTeX leírások a www.latex.lap.hu weboldalon vannak.
%

\documentclass[12pt]{report}

%Magyar nyelvi támogatás (Babel 3.7 vagy későbbi kell!)
\def\magyarOptions{defaults=hu-min}
\usepackage[magyar]{babel}

%Az ékezetes betűk használatához:
\usepackage{t1enc}% ékezetes szavak automatikus elválasztásához
\usepackage[utf8]{inputenc}% ékezetes szavak beviteléhez

% A formai kovetelmenyekben megkövetelt Times betűtípus használata:
\usepackage{times}

%Az AMS csomagjai
\usepackage{amsmath}
\usepackage{amssymb}
\usepackage{amsthm}

%A fejléc láblécek kialakításához:
\usepackage{fancyhdr}

%Természetesen további csomagok is használhatók,
%például ábrák beillesztéséhez a graphix és a psfrag,
%ha nincs rájuk szükség természetesen kihagyhatók.
\usepackage{graphicx}
\usepackage{psfrag}
%_ \usepackage{stackengine,trimclip}
%_ \usepackage{subcaption}

\usepackage{setspace}

\usepackage{hyperref}
\hypersetup{
    pdfauthor={KÉRI Tamás},
    pdftitle={Fuzz tesztelő keretrendszer bűvítése webes felülettel},
    pdfsubject={BSc szakdolgozat},
    pdfkeywords={fuzz testing, web ui, tornado},
    colorlinks,
    citecolor=black,
    filecolor=black,
    linkcolor=black,
    urlcolor=black
}

\usepackage{cite}

\usepackage{enumerate}
\usepackage{enumitem}

\setlist[enumerate]{%
    topsep=1ex,%
    itemsep=-1ex,%
    partopsep=1ex,%
    parsep=1ex,%
}
\setlist[itemize]{%
    noitemsep,%
    topsep=1ex,%
}
\setlist[description]{%
    noitemsep,%
}

% Színek
\usepackage{color}
\definecolor{myCommentColorGreen}{RGB}{0,128,0}
\definecolor{myLineNumberGray}{RGB}{128,128,128}

% Forráskódhoz
\usepackage{listings}
\usepackage{lstautogobble}
\lstdefinelanguage{JavaScript}{
  keywords={break, case, catch, continue, debugger, default, delete, do, else,%
            finally, for, function, if, in, instanceof, new, return, switch,%
            this, throw, try, typeof, var, void, while, with},
  morecomment=[l]{//},
  morecomment=[s]{/*}{*/},
  morestring=[b]',
  morestring=[b]",
  sensitive=true
}
\lstset{ %
    %backgroundcolor=\color{white},   % choose the background color; you must add  \usepackage{color} or \usepackage{xcolor}
    basicstyle=\scriptsize, %\footnotesize,         % the size of the fonts that are used for    the code
    %breakatwhitespace=false,         % sets if automatic breaks should only      happen at whitespace
    breaklines=true,                  % sets automatic line breaking
    captionpos=b,                    % sets the caption-position to          bottom
    commentstyle=\color{myCommentColorGreen},    % comment style
    %deletekeywords={...},            % if you want to delete keywords              from the given language
    %escapeinside={\%*}{*)},          % if you want to add LaTeX                within your code
    %extendedchars=true,              % lets you use non-ASCII                  characters; for 8-bits encodings only, does not work with                  UTF-8
    frame=leftline,                   % adds a frame around the                    code
    %keepspaces=true,                 % keeps spaces in text,                      useful for keeping indentation of code (possibly needs                      columns=flexible)
    inputpath=../../../,
    keywordstyle=\color{blue},        % keyword style
    language=C++,                     % the language of                          the code
    literate=%
        {á}{{\'a}}1
        {é}{{\'e}}1
        {í}{{\'i}}1
        {ó}{{\'o}}1
        {ö}{{\"o}}1
        {ő}{{\H{o}}}1
        {ú}{{\'u}}1
        {ü}{{\"u}}1
        {ű}{{\H{u}}}1
        {Á}{{\'A}}1
        {É}{{\'E}}1
        {Í}{{\'I}}1
        {Ó}{{\'O}}1
        {Ö}{{\"O}}1
        {Ő}{{\H{O}}}1
        {Ú}{{\'U}}1
        {Ü}{{\"U}}1
        {Ű}{{\H{U}}}1,
    %morekeywords={*,...},            % if you want to                            add more keywords to the set
    numberfirstline=true,
    numbers=left,                     % where to put                              the line-numbers; possible values are (none, left, right)
    numbersep=10pt,                   % how far theline-numbers are from the code
    numberstyle=\tiny\color{myLineNumberGray}, % the stylethat is used for the line-numbers
    %name=\thelstnumber,              %
    %rulecolor=\color{black},         % if notset, the frame-color may be changed online-breaks within not-black text (e.g.comments (green here))
    showspaces=false,                % showspaces everywhere adding particularunderscores; it overrides'showstringspaces'
    showstringspaces=false,          % underline spaces within strings only
    showtabs=false,                  % show tabs within strings addingparticular underscores
    %stepnumber=2,                    % the step between two line-numbers.If it's 1, each line will benumbered
    %stringstyle=\color{mymauve},     % string literal style
    tabsize=4,                        % sets default tabsize to 2    spaces
    %title=\lstname                   % show the filename of     files included with     \lstinputlisting; also try     caption instead of title
}

% Kép mellett folyó íráshoz
\usepackage{wrapfig}

\usepackage[labelfont=it,justification=centering]{caption}

\usepackage{epigraph}
%\epigraphfontsize{\small\itshape}

\usepackage[titletoc,title]{appendix}
\usepackage{multirow}
\usepackage{threeparttable}
%\usepackage{numprint}
%_ \usepackage{siunitx}

%Tételszerű környezetek definiálhatók, ezek most fejezetenként együtt számozódnak, pl.
\newtheorem{tét}{Tétel}[chapter]
\newtheorem{defi}[tét]{Definíció}
\newtheorem{lemma}[tét]{Lemma}
\newtheorem{áll}[tét]{Állítás}
\newtheorem{köv}[tét]{Következmény}

%Ha a megjegyzések és a példak szövegét nem akarjuk dőlten szedni, akkor
%az alábbi parancs után kell őket definiální:
\theoremstyle{definition}
\newtheorem{megj}[tét]{Megjegyzés}
\newtheorem{pld}[tét]{Példa}

%%% Saját parancsok %%%%%%%%%%%%%%%%%%%%%%%%%%%%%%%%%%%%%%%%%%%%%%%%%%

%%% Az angol kifejezések kiemelése.
\newcommand{\inenglish}[1]{\textsl{#1}}
\newcommand{\inenglishfn}[1]{\footnotesize{\inenglish{#1}}}

%%% A függvények a szövegben.
\newcommand{\func}[1]{{\textsl{#1}}}

%_ %%% Globális beállítás a WRAPFIGUREnak.
%_ % A WRAPFIGUREnak adhatjuk meg az egész dolgozatban, hogy milyen globális
%_ % stratégia szerint helyezze el a képeket: r - right, o - outside, l - left,
%_ % i - inside, stb. A magyar makrónév melletti érv a lehetséges ütközések
%_ % elkerülése. Az alapbeállítás az 'r', a szakdolgozat egyoldalas
%_ % nyomtatása miatt, ha azonban az aktuális helyzet megkívánja, a makró
%_ % használatát mellőzzük.
%_ \newcommand{\melyikoldalra}{r}
%_ %\newcommand{\melyikoldalra}{o} % az oldal külső felére
%_ 
%_ %%% Az INCLUDEGRAPHICS ,,resolution'' kulcsszavának definíciója.
%_ % resolution=<res in DPI>
%_ %\makeatletter
%_ %\define@key{Gin}{resolution}{\pdfimageresolution=#1\relax}
%_ %\makeatother
%_ 
%_ %%% Az INCLUDEGRAPHICS kulcsszavainak globális beállításai.
%_ %\setkeys{Gin}{resolution=2400}
%_ 
%_ %%% Az INCLUDEGRAPHICS kivágások globális eltolása.
%_ % A néhány sorral lentebbi \setpdfoffset{0pt}{0pt} átállításával a
%_ % a kivágást globálisan eltolhatjuk.
%_ \makeatletter
%_ \define@key{Gin}{xviewport}[]{%
%_   \begingroup\edef\x{%
%_     \endgroup\noexpand\setkeys{Gin}{viewport=\x@viewport}%
%_   }\x
%_ }
%_ \newlength{\Xoffset}
%_ \newlength{\Yoffset}
%_ \newcommand*{\setpdfoffset}[2]{%
%_   \setlength{\Xoffset}{#1}%
%_   \setlength{\Yoffset}{#2}%
%_ }
%_ \newcommand*{\setviewport}[4]{%
%_   \def\x@viewport{%
%_     {\the\dimexpr#1-\Xoffset}
%_     {\the\dimexpr#2-\Yoffset}
%_     {\the\dimexpr#3-\Xoffset}
%_     {\the\dimexpr#4-\Yoffset}%
%_   }%
%_ }
%_ \makeatother
%_ % Paraméterek:
%_ %   #1 - llx, alsó bal sarok x érték,
%_ %   #2 - lly, alsó bal sarok y érték,
%_ %   #3 - urx, felső jobb sarok x érték,
%_ %   #4 - ury, felső jobb sarok y érték,
%_ %   #5 - szabad lokális beállítások,
%_ %   #6 - a beszúrandó kép.
%_ \newcommand{\includegraphicskivagas}[6]{
%_     \setpdfoffset{0pt}{0pt}
%_     \setviewport{#1}{#2}{#3}{#4}
%_     \includegraphics[xviewport,clip,#5]{#6}
%_ }
%_ % Az 'img/built/dataflow_eps' képre specializált INCLUDEGRAPHICSKIVAGAS.
%_ \newcommand{\includedataflowkivagas}[5]{
%_     \includegraphicskivagas{#1}{#2}{#3}{#4}{scale=0.6,#5}
%_     {img/built/dataflow_eps}
%_ }

%\hyphenation{sza-bály el-kü-lö-ní-tett}

%_ \newcommand{\komment}[1]{\iffalse{#1}\fi}

%_ %%% A lábjegyzeteket nem töri át másik oldalra
%_ \interfootnotelinepenalty=10000

%%% Saját parancsok %%% VÉGE %%%%%%%%%%%%%%%%%%%%%%%%%%%%%%%%%%%%%%%%%

%Margók:
\hoffset -1in
\voffset -1in
\oddsidemargin 35mm
\textwidth 150mm
\topmargin 15mm
\headheight 10mm
\headsep 5mm
\textheight 237mm

\begin{document}


%%%%%%%%%%%%%%%%%%%%%%%%%%%%%%%%%%%%%%%%%%%%%%%%%%%%%%%%%%%%%%%%%%%%%%
%%   Címlap                                                         %%
%%%%%%%%%%%%%%%%%%%%%%%%%%%%%%%%%%%%%%%%%%%%%%%%%%%%%%%%%%%%%%%%%%%%%%

    %A FEJEZETEK KEZDŐOLDALAINAK FEJ ÉS LÁBLÉCE:
    %a plain oldalstílust kell átdefiniálni, hogy ott ne legyen fejléc:
    \fancypagestyle{plain}{%
    %ez mindent töröl:
    \fancyhf{}
    % a láblécbe jobboldalra kerüljön az oldalszám:
    \fancyhead[R]{\thepage}
    %elválasztó vonal sem kell:
    \renewcommand{\headrulewidth}{0pt}
    }

    %A TÖBBI OLDAL FEJ ÉS LÁBLÉCE:
    \pagestyle{fancy}
    \fancyhf{}
    \fancyhead[L]{\rightmark}
    \fancyhead[R]{\thepage}


    %A címoldalra se fej- se lábléc nem kell:
    \thispagestyle{empty}

    \begin{center}
    \vspace*{1cm}
    {\Large\bf Szegedi Tudományegyetem}

    \vspace{0.5cm}

    {\Large\bf Informatikai Intézet}

    \vspace*{3.8cm}

    % Tíz sorral fentebb is át kell írni!!!
    {\LARGE\bf Fuzz tesztelő keretrendszer bűvítése webes felülettel}


    \vspace*{3.6cm}

    %{\Large Diplomamunka}
    {\Large Szakdolgozat}

    \vspace*{4cm}

    %Értelemszerűen megváltoztatandó:
    {\large
    \begin{tabular}{c@{\hspace{4cm}}c}
    \emph{Készítette:}     &\emph{Témavezető:}\\
    \bf{Kéri Tamás József}     &\bf{Dr. Kiss Ákos}\\
    informatika szakos     &egyetemi adjunktus\\
    hallgató &
    \end{tabular}
    }

    \vspace*{2.3cm}

    {\Large
    Szeged
    \\
    \vspace{2mm}
    2019
    }
    \end{center}


    % 1.5-ös sorköz:
    % ezt javasolják:  \linespread{1.25}
    % és ez bevált, de ehhez kellett a \usepackage{setspace} csomag betöltése.
    \onehalfspacing


%_ %%%%%%%%%%%%%%%%%%%%%%%%%%%%%%%%%%%%%%%%%%%%%%%%%%%%%%%%%%%%%%%%%%%%%%
%_ %%   Mottó                                                          %%
%_ %%%%%%%%%%%%%%%%%%%%%%%%%%%%%%%%%%%%%%%%%%%%%%%%%%%%%%%%%%%%%%%%%%%%%%
%_ 
%_     \clearpage
%_     \thispagestyle{empty}
%_     {
%_     \setlength\epigraphrule{0pt}
%_     \linespread{1.0}\epigraph{\small\emph{,,A világ csak híd, menj át rajta,
%_     de ne építs rajta házat.''}}{\small(IHS)}
%_     }
%_ 

%%%%%%%%%%%%%%%%%%%%%%%%%%%%%%%%%%%%%%%%%%%%%%%%%%%%%%%%%%%%%%%%%%%%%%
%%   Tartalomjegyzék                                                %%
%%%%%%%%%%%%%%%%%%%%%%%%%%%%%%%%%%%%%%%%%%%%%%%%%%%%%%%%%%%%%%%%%%%%%%

    \tableofcontents


%%%%%%%%%%%%%%%%%%%%%%%%%%%%%%%%%%%%%%%%%%%%%%%%%%%%%%%%%%%%%%%%%%%%%%
%%   Feladatkiírás                                                  %%
%%%%%%%%%%%%%%%%%%%%%%%%%%%%%%%%%%%%%%%%%%%%%%%%%%%%%%%%%%%%%%%%%%%%%%


    %A \chapter* parancs nem ad a fejezetnek sorszámot
    \chapter*{Feladatkiírás}
    %A tartalomjegyzékben mégis szerepeltetni kell, mint szakasz(section) szerepeljen:
    \addcontentsline{toc}{section}{Feladatkiírás}

Elméletileg Ákos adja meg, de inkább elecroval dumáljátok le...


%%%%%%%%%%%%%%%%%%%%%%%%%%%%%%%%%%%%%%%%%%%%%%%%%%%%%%%%%%%%%%%%%%%%%%
%%   Tartalmi összefoglaló                                          %%
%%%%%%%%%%%%%%%%%%%%%%%%%%%%%%%%%%%%%%%%%%%%%%%%%%%%%%%%%%%%%%%%%%%%%%

    \chapter*{Tartalmi összefoglaló}
    \addcontentsline{toc}{section}{Tartalmi összefoglaló}

%A tartalmi összefoglalónak tartalmaznia kell (rövid, legfeljebb egy oldalas, összefüggő megfogalmazásban)
%a következőket: a téma megnevezése, a megadott feladat megfogalmazása - a feladatkiíráshoz viszonyítva-,
%a megoldási mód, az alkalmazott eszközök, módszerek, az elért eredmények, kulcsszavak (4-6 darab).

%Az összefoglaló nyelvének meg kell egyeznie a dolgozat nyelvével. Ha a dolgozat idegen nyelven készül,
%magyar nyelvű tartalmi összefoglaló készítése is kötelező (külön lapon), melynek terjedelmét a TVSZ szabályozza.

    \subsubsection*{A téma megnevezése}

ABC
%_ A \emph{World Wide Web Consortium} (W3C) által \emph{Canvas 2D Context} néven
%_ 2015-ben elfogadott ajánlás \emph{Path API} részének megvalósítása
%_ \emph{OpenGL~ES~2.0} alapokon.

    \subsubsection*{A megadott feladat megfogalmazása}

ABC
%_ A \emph{World Wide Web Consortium} (W3C) által \emph{Canvas 2D Context} néven
%_ 2015-ben elfogadott ajánlás \emph{Path API} részének megvalósítása
%_ \emph{OpenGL~ES~2.0} alapokon.

    \subsubsection*{A megoldásmód}

ABC ez majd t9r9lni kell \cite{Herczeg:2014:GL2D}
%_ A \emph{World Wide Web Consortium} (W3C) által \emph{Canvas 2D Context} néven
%_ 2015-ben elfogadott ajánlás \emph{Path API} részének megvalósítása
%_ \emph{OpenGL~ES~2.0} alapokon.

    \subsubsection*{Alkalmazott eszközök, módszerek}


    \subsubsection*{Elért eredmények}


    \subsubsection*{Kulcsszavak}

%_  Canvas~2D~Context, OpenGL~ES~2.0, számítógépes grafika (\inenglish{computer
%_  graphics}), szabad alakzat (\inenglish{path}), trapéz tesszelláció
%_  (\inenglish{trapezoid tessellation})
%_ 

%%%%%%%%%%%%%%%%%%%%%%%%%%%%%%%%%%%%%%%%%%%%%%%%%%%%%%%%%%%%%%%%%%%%%%
%%   Bevezetés                                                      %%
%%%%%%%%%%%%%%%%%%%%%%%%%%%%%%%%%%%%%%%%%%%%%%%%%%%%%%%%%%%%%%%%%%%%%%

    \chapter*{Bevezetés}
    \label{sec:Bevezetés}
    \addcontentsline{toc}{section}{Bevezetés}


%%%%%%%%%%%%%%%%%%%%%%%%%%%%%%%%%%%%%%%%%%%%%%%%%%%%%%%%%%%%%%%%%%%%%%
%%   Háttér                                                         %%
%%%%%%%%%%%%%%%%%%%%%%%%%%%%%%%%%%%%%%%%%%%%%%%%%%%%%%%%%%%%%%%%%%%%%%

    \chapter{Háttér}
    \label{sec:Hatter}


    \section{Fuzzinator}
    \label{sec:Fuzzinator}


    \section{Tornado}
    \label{sec:Tornado}


%%%%%%%%%%%%%%%%%%%%%%%%%%%%%%%%%%%%%%%%%%%%%%%%%%%%%%%%%%%%%%%%%%%%%%
%%   ...                                                %%
%%%%%%%%%%%%%%%%%%%%%%%%%%%%%%%%%%%%%%%%%%%%%%%%%%%%%%%%%%%%%%%%%%%%%%

    \chapter{A WUI interfész}


%%%%%%%%%%%%%%%%%%%%%%%%%%%%%%%%%%%%%%%%%%%%%%%%%%%%%%%%%%%%%%%%%%%%%%
%%   Használat és eredmények                                        %%
%%%%%%%%%%%%%%%%%%%%%%%%%%%%%%%%%%%%%%%%%%%%%%%%%%%%%%%%%%%%%%%%%%%%%%

    \chapter{Használat és eredmények}
    %\addcontentsline{toc}{section}{Használat és eredmények}


    \section{Használat}

    \section{Eredmények}
    \label{sec:Eredmények}


%%%%%%%%%%%%%%%%%%%%%%%%%%%%%%%%%%%%%%%%%%%%%%%%%%%%%%%%%%%%%%%%%%%%%%
%%   Összefoglalás                                                  %%
%%%%%%%%%%%%%%%%%%%%%%%%%%%%%%%%%%%%%%%%%%%%%%%%%%%%%%%%%%%%%%%%%%%%%%

    \chapter{Összefoglalás}
    %\addcontentsline{toc}{section}{Összefoglalás}


%%%%%%%%%%%%%%%%%%%%%%%%%%%%%%%%%%%%%%%%%%%%%%%%%%%%%%%%%%%%%%%%%%%%%%
%%   Irodalomjegyzék                                                %%
%%%%%%%%%%%%%%%%%%%%%%%%%%%%%%%%%%%%%%%%%%%%%%%%%%%%%%%%%%%%%%%%%%%%%%

  \bibliography{src/bib/cites}{}
  \bibliographystyle{src/bib/huplain}
  \addtocontents{toc}{\ }
  \addcontentsline{toc}{section}{Irodalomjegyzék}


%%%%%%%%%%%%%%%%%%%%%%%%%%%%%%%%%%%%%%%%%%%%%%%%%%%%%%%%%%%%%%%%%%%%%%
%%   Mellékletek                                                    %%
%%%%%%%%%%%%%%%%%%%%%%%%%%%%%%%%%%%%%%%%%%%%%%%%%%%%%%%%%%%%%%%%%%%%%%

    \chapter*{Mellékletek}
    \newcommand{\thischaptertitle}{Mellékletek}
    \addcontentsline{toc}{section}{Mellékletek}

%%%%%%%%%%%%%%%%%%%%%%%%%%%%%%%%%%%%%%%%%%%%%%%%%%%%%%%%%%%%%%%%%%%%%%
%%   Nyilatkozat                                                    %%
%%%%%%%%%%%%%%%%%%%%%%%%%%%%%%%%%%%%%%%%%%%%%%%%%%%%%%%%%%%%%%%%%%%%%%

    \chapter*{Nyilatkozat}
    %Egy üres sort adunk a tartalomjegyzékhez:
    \addcontentsline{toc}{section}{Nyilatkozat}
    %\hspace{\parindent}

    % A nyilatkozat szövege más titkos és nem titkos dolgozatok esetében.
    % Csak az egyik típusú nyilatkozatnak kell a dolgozatban szerepelni
    % A pontok helyére az adatok értelemszerűen behelyettesítendők és
    % a szakdolgozat /diplomamunka szó megfelelően kiválasztandó.


    % A nyilatkozat szövege TITKOSNAK NEM MINŐSÍTETT dolgozatban a következő:
    % A pontokkal jelölt szövegrészek értelemszerűen a szövegszerkesztőben és
    % nem kézzel helyettesítendők:

    \noindent

Alulírott
%\makebox[4cm]{\dotfill}
\textbf{Kéri Tamás József}, \textit{programtervező informatikus}
szakos hallgató, kijelentem, hogy a dolgozatomat a \textit{Szegedi
Tudományegyetem},
\textit{Informatikai Tanszékcsoport
Szoftverfejlesztés
%\makebox[4cm]{\dotfill}
Tanszékén} készítettem,
\textit{programtervező informatikus BSc.}
%\makebox[4cm]{\dotfill}
diploma megszerzése érdekében.

Kijelentem, hogy a dolgozatot más szakon korábban nem védtem meg, saját
munkám eredménye, és csak a hivatkozott forrásokat (szakirodalom,
eszközök, stb.) használtam fel.

Tudomásul veszem, hogy szakdolgozatomat
% / diplomamunkámat
a Szegedi Tudományegyetem Informatikai Tanszékcsoport könyvtárában, a
helyben olvasható könyvek között helyezik el.

    \vspace*{2cm}

    \begin{tabular}{lc}
    Szeged, \today\
    \hspace{2cm} & \makebox[6cm]{\dotfill} \\
    & aláírás \\
    \end{tabular}


%%%%%%%%%%%%%%%%%%%%%%%%%%%%%%%%%%%%%%%%%%%%%%%%%%%%%%%%%%%%%%%%%%%%%%
%%   Köszönetnyilvánítás                                            %%
%%%%%%%%%%%%%%%%%%%%%%%%%%%%%%%%%%%%%%%%%%%%%%%%%%%%%%%%%%%%%%%%%%%%%%

    \chapter*{Köszönetnyilvánítás}
    \addcontentsline{toc}{section}{Köszönetnyilvánítás}


\end{document}
